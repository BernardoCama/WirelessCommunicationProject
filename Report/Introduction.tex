In this section, we describe briefly the key concepts that 
are mandatory for the comprehension of the project.

\subsection{OFDM signal}

OFDM (Orthogonal Frequency Division Multiplexing)
is a multi-carrier modulation whose key feautures are:
\begin{enumerate}
    \item \textbf{Flexibilty} By means of:
    \begin{itemize}
        \item   \textbf{Adaptive Bit Loading} Adaptive modulation, coding for each sub-carrier.
        \item   \textbf{Multiple Access} Multiple Access Feature with the use of OFDMA.
    \end{itemize}
    \item \textbf{Digital Implementation} With the use of DFT and IDFT at Tx and Rx side respectively to pass from Samples in frequency domain to time domain and viceversa.
    \item \textbf{Simple Equalization} Through the use of Cyclic Prefix that permits the representation of the channel with a single tap (Flat Ch) in each sub-carrier.
    \item \textbf{MIMO Implemetaion} Suited for the use of MIMO/MMIMO Systems.
\end{enumerate}

The main parameters of the Modulation are:
\begin{enumerate}
    \item \textbf{Nsc} Number of sub-carrier which usually is in the form $2^{b}$ since it is optimized for the FFT and IFFT implementaion.\\
                        In our case: Nsc = 64.
    \item \textbf{CyclicPrefixLength} The length of Cyclic Prefix that must be much smaller than Ts (OFDM Symbol Time).\\
                        In our case: CyclicPrefixLength = 4.
    \item \textbf{NumGuardBandCarriers} The number of Guard Bands in frequency domain to protect the OFDM Spectrum from other adjacent trasmissions.\\
                        In our case: NumGuardBandCarriers = [1;1] one for each side.
    \item \textbf{Pilot Positions} The sub-carrier location of Pilot signals, known sequence of symbols that are used to estimate tha channel.\\
                        In our case: Pilotindices = [2]. We really don't use use the Pilot, since we consider the first symbols of the transmission as known, but for completeness we dropped off at least one Pilot.
\end{enumerate}


\subsection{3GPP Standard}

3GPP (3rd Generation Partnership Project) was founded in December 1998 when the European 
Telecommunications Standards Institute (ETSI) 
partnered with other standard development organizations (SDOs) 
from around the world to develop new technologies (technology specifications).

As Channel Model we use QuaDRiGa (QUAsi Deterministic RadIo channel GenerAtor) that 
generates realistic radio channel impulse responses for system-level simulations of mobile radio networks. 
Quadriga indeed is able to simulate 3GPP channel models like 3GPP-3D and also the latest New Radio channel model.


\subsection{Beamforming}

A beamformer can be considered a spatial filter that suppresses
the signal from all directions, except the desired ones by means 
of weights applied to the signals coming from the single array elements;
resulting in controlling the radiation pattern of the array.

The \textbf{Array Pattern Function} $AF(\theta,\phi)$ is the gain that we can obtain with the 
Beamformer in a given direction specified by:

\begin{itemize}
    \item $\theta$ = Elevation ($\pi/2$ - zenith of arrival)
    \item $\phi$ = AoA (angle of arrival)
\end{itemize}

and it is defined as: \\\\
$AF(\theta,\phi)$ = $w^Hs$ where:
\begin{itemize}
    \item $s$ are the Steering Vectors
    \item $w$ are the weigths of the Beamformer
\end{itemize}

In the project, in particular in \hyperref[sec:antenna_array_comparison]{section 4}, we compare the different \textbf{Array Pattern Function} 
of various type of Beamformer.
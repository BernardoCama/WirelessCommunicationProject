In this section, we describe the structure of out project with a brief description of 
the main points of all the parts composing our work.

\subsection{Beamforming techniques}

We have implemented 5 beamforming techniques:

\begin{enumerate}
    \item \textbf{Simple beamforming} The phases are selected to steer the array in a particular direction.
    \item \textbf{Null-steering beamforming} Used to cancel K $\leq$ N-2 plane waves arriving from known directions (N = number of
                antennas).
    \item \textbf{Minimun variance distiortionless response (MVDR) beamforming} This beamformer minimizes the interference-plus-noise 
                power at the output of the beamformer. 
    \item \textbf{Minimum mean square error (MMSE) beamforming} The weigths of the antennas are adjusted in a way that the
            MSE between the output of the beamformer and the reference signal is minimized.
    \item \textbf{Least mean square (LMS) beamforming} This iterative algorithms adjust the weights by estimating the
            gradient of the MSE and moving them in the negative direction of the gradient at each iteration. We have 
            implemented this iterative algorithm both in the time and in the frequency domain.
\end{enumerate}

\subsection{Channels}

All the 5 beamformers have been tested on 3 different channels:

\begin{enumerate}
    \item \textbf{LOS channel} A simple line of sigth channel with no reflections.
    \item \textbf{Two-ray channel} For each signal, we consider a direct path (LOS) and a single reflection.
    \item \textbf{QuaDRiGa channel} Here, the scenario we have used is the $QuaDRiGa\_UD2D\_LOS$.
\end{enumerate}

\subsection{Signals}

For all the beamformers and in all the channel, the bits we transmit are generated randomly and modulated first with a 4\-QAM 
modulation (we have also tested the beamformers with a 16\-QAM); then, the QAM symbols are modulated with OFDM for transmission.

\subsection{Reported results}

In this report, we only describe the three most important simulations we have done:

\begin{enumerate}
    \item Comparison between the SNR at the input and at the output of all the five beamformers in a LOS channel (\hyperref[sec:snr_comparison]{section 3}).
    \item Comparison of the performance in terms of constellations revealed, array pattern functions of the antenna arrya and BER for all the \\ 
            five beamformers considering different antenna arrays (\hyperref[sec:antenna_array_comparison]{section 4}). This has been done in the quadriga channel
            and using the LMS beamformer in the time domain.
    \item Traking of two vehicles using LMS beamforming in the frequency domain (\hyperref[sec:tracking]{section 5}).
\end{enumerate}

All the other simulations we have done can be found on the \textit{Github} repository of the project (link in title page).